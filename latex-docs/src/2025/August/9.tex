\subsection{9. August 2025 - Nginx Ingress Controller Setup}

Dieser Abschnitt dokumentiert die Installation und Konfiguration des Nginx Ingress Controllers für K3s mittels Helm. Der Ingress Controller ermöglicht HTTP/HTTPS-Routing zu verschiedenen Services innerhalb des Kubernetes-Clusters.

\subsubsection{Helm Installation des Nginx Ingress Controllers}

Die Installation erfolgt über das offizielle Helm Chart gemäß der Kubernetes Ingress-Nginx Dokumentation:

\textbf{Repository hinzufügen und Installation:}

\begin{verbatim}
helm repo add ingress-nginx https://kubernetes.github.io/ingress-nginx
helm repo update
helm install ingress-nginx ingress-nginx/ingress-nginx \
  --namespace ingress-nginx --create-namespace
\end{verbatim}

Referenz: \texttt{https://kubernetes.github.io/ingress-nginx/deploy/}

\subsubsection{Deployment-Status und Verifikation}

Nach erfolgreicher Installation wird folgende Bestätigung ausgegeben:

\begin{verbatim}
Release "ingress-nginx" does not exist. Installing it now.
NAME: ingress-nginx
LAST DEPLOYED: Sat Aug  9 15:45:49 2025
NAMESPACE: ingress-nginx
STATUS: deployed
REVISION: 1
TEST SUITE: None
NOTES:
The ingress-nginx controller has been installed.
It may take a few minutes for the load balancer IP to be available.
You can watch the status by running 'kubectl get service 
--namespace ingress-nginx ingress-nginx-controller 
--output wide --watch'
\end{verbatim}

\subsubsection{Ingress-Ressourcen Konfiguration}

\paragraph{Standard Ingress-Beispiel}

Die folgende YAML-Konfiguration demonstriert eine grundlegende Ingress-Ressource:

\begin{verbatim}
apiVersion: networking.k8s.io/v1
kind: Ingress
metadata:
  name: example
  namespace: foo
spec:
  ingressClassName: nginx
  rules:
    - host: www.example.com
      http:
        paths:
          - pathType: Prefix
            backend:
              service:
                name: exampleService
                port:
                  number: 80
            path: /
  # TLS-Konfiguration (optional)
  tls:
    - hosts:
      - www.example.com
      secretName: example-tls
\end{verbatim}

\paragraph{TLS-Secret Konfiguration}

Für HTTPS-Unterstützung muss ein entsprechendes Secret mit Zertifikat und privatem Schlüssel erstellt werden:

\begin{verbatim}
apiVersion: v1
kind: Secret
metadata:
  name: example-tls
  namespace: foo
data:
  tls.crt: <base64 encoded cert>
  tls.key: <base64 encoded key>
type: kubernetes.io/tls
\end{verbatim}

\subsubsection{Service-Status Überwachung}

Der LoadBalancer-Status kann kontinuierlich überwacht werden:

\begin{verbatim}
kubectl get service --namespace ingress-nginx \
  ingress-nginx-controller --output wide --watch
\end{verbatim}

Der Nginx Ingress Controller ist nun einsatzbereit und ermöglicht flexibles HTTP/HTTPS-Routing zu verschiedenen Cluster-Services mit automatischer SSL-Terminierung.
