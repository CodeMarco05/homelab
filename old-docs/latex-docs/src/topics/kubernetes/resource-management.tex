\subsection{K3s Master Node Ressourcen-Management}

Dieser Abschnitt dokumentiert die Implementierung von systemd-basierten Ressourcenbeschränkungen für den K3s Master Node zur Vermeidung von Ressourcen-Überbelegung und zur Gewährleistung stabiler Cluster-Performance.

\subsubsection{Problemstellung und Zielsetzung}

\paragraph{Ressourcen-Herausforderungen im Homelab}

K3s Master Nodes in ressourcenbeschränkten Homelab-Umgebungen erfordern präzise Ressourcenkontrolle:

\begin{itemize}
  \item \textbf{Memory Pressure} - Unkontrollierte Speichernutzung kann System-Instabilität verursachen
  \item \textbf{CPU Überlastung} - Übermäßige CPU-Nutzung beeinträchtigt Control Plane Performance
  \item \textbf{System Responsiveness} - Host-System muss für administrative Aufgaben verfügbar bleiben
  \item \textbf{Resource Contention} - Andere Services benötigen garantierte Ressourcen-Reservierung
\end{itemize}

\paragraph{Zielsetzung der Ressourcenbeschränkung}

\begin{itemize}
  \item Stabile K3s Control Plane Operation bei begrenzten Ressourcen
  \item Vermeidung von OOM-Killer-Ereignissen durch Memory-Limits
  \item CPU-Throttling zur Erhaltung der System-Responsiveness
  \item Proaktive Ressourcen-Überwachung und -Steuerung
\end{itemize}

\subsubsection{systemd Service Unit Konfiguration}

\paragraph{Erweiterte K3s Service Unit}

Die systemd Service Unit wird um Ressourcenbeschränkungen für 8GB RAM Server erweitert:

\begin{verbatim}
# File: /etc/systemd/system/k3s.service

[Unit]
Description=Lightweight Kubernetes
Documentation=https://k3s.io
Wants=network-online.target
After=network-online.target

[Install]
WantedBy=multi-user.target

[Service]
Type=notify
EnvironmentFile=-/etc/default/%N
EnvironmentFile=-/etc/sysconfig/%N
EnvironmentFile=-/etc/systemd/system/k3s.service.env
KillMode=process
Delegate=yes
User=root

# Standard Limits für Container-Performance
LimitNOFILE=1048576
LimitNPROC=infinity
LimitCORE=infinity
TasksMax=infinity

# Ressourcenbeschränkungen für 8GB RAM Server
MemoryMax=6963M       # Maximaler RAM-Verbrauch (~85% von 8GB)
MemoryHigh=6553M      # High-Water-Mark für Memory Pressure \
                      # (~80% von 8GB)
CPUQuota=85%          # CPU-Limitierung auf 85% aller verfügbaren Cores

TimeoutStartSec=0
Restart=always
RestartSec=5s

ExecStartPre=/bin/sh -xc \
  '! /usr/bin/systemctl is-enabled --quiet \
    nm-cloud-setup.service 2>/dev/null'
ExecStartPre=-/sbin/modprobe br_netfilter
ExecStartPre=-/sbin/modprobe overlay

ExecStart=/usr/local/bin/k3s \
    server \
    '--node-name=master' \
    '--write-kubeconfig-mode=644' \
    '--node-ip=<VPN_IP>' \
    '--advertise-address=<VPN_IP>' \
    '--node-external-ip=<PUBLIC_IP>' \
    '--flannel-iface=tailscale0' \
    '--tls-san=<VPN_IP>' \
    '--tls-san=<PUBLIC_IP>' \
    '--tls-san=localhost' \
    '--tls-san=127.0.0.1'
\end{verbatim}

\subsubsection{Ressourcen-Parameter Erklärung}

\paragraph{Memory Management Parameter}

\textbf{MemoryMax=6963M (Harte Memory-Grenze):}
\begin{itemize}
  \item Absolute Obergrenze für RAM-Verbrauch des K3s Service
  \item Berechnung: 8GB x 0.85 = 6.8GB ca. 6963MB
  \item Verhindert OOM-Killer-Ereignisse auf System-Ebene
  \item Reserviert ~1.2GB für Host-System und andere Services
\end{itemize}

\textbf{MemoryHigh=6553M (Soft Memory-Limit):}
\begin{itemize}
  \item Auslöser für proaktive Memory-Bereinigung
  \item Berechnung: 8GB x 0.80 = 6.4GB ca. 6553MB
  \item Aktiviert Kernel Memory Reclaim vor Erreichen der harten Grenze
  \item Reduziert Memory Pressure durch präventive Garbage Collection
\end{itemize}

\paragraph{CPU Management Parameter}

\textbf{CPUQuota=85\%:}
\begin{itemize}
  \item Beschränkung auf 85\% aller verfügbaren CPU-Cores
  \item Beispiel: 4-Core System -> 3.4 Cores für K3s verfügbar
  \item Gewährleistet CPU-Reservierung für Host-System-Aufgaben
  \item Verhindert vollständige CPU-Sättigung durch Kubernetes Workloads
\end{itemize}

\subsubsection{Implementierung und Aktivierung}

\paragraph{Service Unit Deployment}

\begin{verbatim}
# Bestehenden K3s Service stoppen
sudo systemctl stop k3s

# Service Unit aktualisieren
sudo cp k3s.service /etc/systemd/system/k3s.service

# systemd Daemon neu laden
sudo systemctl daemon-reload

# Service Unit validieren
sudo systemctl cat k3s.service

# K3s Service mit neuen Limits starten
sudo systemctl start k3s
sudo systemctl enable k3s
\end{verbatim}

\paragraph{Ressourcen-Limit Verifikation}

\begin{verbatim}
# Service Status mit Ressourcen-Information
sudo systemctl status k3s

# Detaillierte Ressourcen-Limits anzeigen
sudo systemctl show k3s | grep -E "(Memory|CPU)"

# Live-Ressourcen-Verbrauch überwachen
sudo systemctl show k3s | grep -E "(MemoryCurrent|CPUUsage)"
\end{verbatim}

\subsubsection{Monitoring und Überwachung}

\paragraph{Ressourcen-Verbrauch Monitoring}

\begin{verbatim}
# Memory-Verbrauch des K3s Service
sudo systemd-cgtop -p -1 | grep k3s

# Detailliertes cgroup Memory-Status
cat /sys/fs/cgroup/system.slice/k3s.service/memory.current
cat /sys/fs/cgroup/system.slice/k3s.service/memory.max
cat /sys/fs/cgroup/system.slice/k3s.service/memory.high

# CPU-Verbrauch und Throttling
cat /sys/fs/cgroup/system.slice/k3s.service/cpu.stat
\end{verbatim}

\subsubsection{Tuning und Anpassungen}

\paragraph{RAM-basierte Konfigurationsempfehlungen}

\textbf{Für 4GB RAM Server:}
\begin{verbatim}
MemoryMax=3276M      # ~80% von 4GB
MemoryHigh=2949M     # ~72% von 4GB
CPUQuota=80%         # Konservativer für begrenzte Ressourcen
\end{verbatim}

\textbf{Für 16GB RAM Server:}
\begin{verbatim}
MemoryMax=13926M     # ~85% von 16GB
MemoryHigh=13107M    # ~80% von 16GB
CPUQuota=90%         # Weniger restriktiv bei mehr Ressourcen
\end{verbatim}

\paragraph{Performance-Optimierung Parameter}

\begin{verbatim}
# Zusätzliche systemd Parameter für Performance
IOWeight=500              # I/O Priorität reduzieren
Nice=5                   # Niedrigere CPU-Priorität
OOMScoreAdjust=100       # OOM-Killer Präferenz erhöhen

# cgroup v2 spezifische Optimierungen
MemorySwapMax=2G         # Swap-Nutzung begrenzen
IODeviceLatencyTargetSec=100ms  # I/O Latenz-Ziel
\end{verbatim}

Diese Ressourcenbeschränkungen gewährleisten einen stabilen K3s Master Node Betrieb in ressourcenbeschränkten Homelab-Umgebungen durch proaktive Ressourcenkontrolle und systemd-basierte Limits.