\subsection{Netzwerk-Architektur mit Tailscale und Nginx Proxy Manager}

Die Homelab-Infrastruktur basiert auf einer mehrstufigen Netzwerk-Architektur, die sichere VPN-Verbindungen mit flexiblem Reverse-Proxy-Management kombiniert.

\subsubsection{Tailscale VPN-Integration}

Jeder Server im Cluster benötigt eine Tailscale-Installation als primäre VPN-Lösung:

\begin{verbatim}
curl -fsSL https://tailscale.com/install.sh | sh
sudo tailscale up
\end{verbatim}

Tailscale ermöglicht:
\begin{itemize}
  \item Sichere Mesh-Netzwerk-Verbindungen zwischen allen Nodes
  \item Automatische Peer-Discovery ohne manuelle Routing-Konfiguration  
  \item Zero-Trust-Netzwerk-Architektur mit End-to-End-Verschlüsselung
  \item Zentrale Zugriffskontrolle über das Tailscale Admin Panel
\end{itemize}

\subsubsection{Nginx Proxy Manager für SSL-Terminierung}

Nginx Proxy Manager (NPM) fungiert als zentraler Reverse Proxy und SSL-Termination-Point für alle Homelab-Services.

\paragraph{Deployment via Docker Compose}

\begin{verbatim}
services:
  nginx-proxy-manager:
    image: 'jc21/nginx-proxy-manager:latest'
    restart: unless-stopped
    ports:
      - '80:80'
      - '443:443'
      - '81:81'
    volumes:
      - ./data:/data
      - ./letsencrypt:/etc/letsencrypt
\end{verbatim}

\paragraph{SSL-Zertifikat-Management mit DNS Challenge}

Für private IP-Adressen und interne Services wird DNS Challenge über Cloudflare verwendet:

\textbf{Cloudflare DNS Challenge Konfiguration:}
\begin{itemize}
  \item Verwendung von Cloudflare API-Token für automatische DNS-Validierung
  \item Unterstützung für Wildcard-Zertifikate (\texttt{*.domain.com})
  \item Automatische Zertifikat-Erneuerung ohne Port-Exposition
  \item SSL-Zertifikate auch für private Tailscale-IPs
\end{itemize}

\paragraph{Proxy-Routing zu Kubernetes Services}

NPM ermöglicht die Weiterleitung zu K3s Services über private Tailscale-Adressen:

\textbf{Beispiel-Routing-Konfiguration:}
\begin{itemize}
  \item \textbf{Domain:} \texttt{app.homelab.com}
  \item \textbf{Ziel:} \texttt{<VPN\_IP>:30080} (Tailscale IP + NodePort)
  \item \textbf{SSL:} DNS Challenge via Cloudflare
  \item \textbf{Zusätzliche Optionen:} WebSocket-Unterstützung, Custom Headers
\end{itemize}

\paragraph{Service-Discovery und Load Balancing}

Die Architektur ermöglicht:
\begin{itemize}
  \item Routing zu verschiedenen K3s Services über eindeutige Subdomains
  \item Load Balancing zwischen mehreren Worker Nodes
  \item Health Checks und automatisches Failover
  \item Zentrale SSL-Zertifikat-Verwaltung für alle Services
\end{itemize}

Diese Konfiguration bietet eine professionelle, skalierbare Lösung für sicheren externen Zugriff auf interne Homelab-Services ohne Kompromisse bei der Netzwerk-Sicherheit.