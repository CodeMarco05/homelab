\subsection{Hardware Setup und ZFS Konfiguration}

Der Server ist nun fertig aufgebaut und läuft. Die Hardware ist in der Anlage zu finden. Momentan ist es so, dass die zwei SSDs in einem Mirror laufen in einem zfs pool. Diesen kann man wie folgt anlegen:

Zuerst habe ich die beiden SSDs formatiert.

\begin{verbatim}
# anzeigen der SSDs
lsblk -o NAME,SIZE,MODEL,TYPE,MOUNTPOINT

# oder
fdisk -l 

# formatieren der Dateisystemsignatur
wipefs -a /dev/sdX

# formatieren des partitionstables
sgdisk --zap-all /dev/sdX
\end{verbatim}

Dann habe ich den ZFS Pool erstellt:

\begin{verbatim}
# ZFS Pool mit Mirror erstellen
sudo zpool create -o ashift=12 homelab-pool mirror /dev/sda /dev/sdb

# Pool Status überprüfen
sudo zpool status

# Pool Eigenschaften anzeigen
sudo zpool list

# Automatic snapshots aktivieren
sudo zfs set com.sun:auto-snapshot=true homelab-pool
\end{verbatim}

\subsubsection{ZFS Dataset Konfiguration}

Für bessere Organisation werden verschiedene Datasets erstellt:

\begin{verbatim}
# Datasets für verschiedene Anwendungen erstellen
sudo zfs create homelab-pool/containers
sudo zfs create homelab-pool/data
sudo zfs create homelab-pool/backups

# Compression aktivieren für bessere Speichereffizienz
sudo zfs set compression=lz4 homelab-pool
sudo zfs set compression=gzip-9 homelab-pool/backups

# Deduplication für Backup-Dataset
sudo zfs set dedup=on homelab-pool/backups
\end{verbatim}

\subsubsection{Monitoring und Wartung}

\begin{verbatim}
# Pool Health Check
sudo zpool status -v

# Scrub für Datenintegrität
sudo zpool scrub homelab-pool

# Scrub Status überprüfen
sudo zpool status | grep scrub

# ZFS Statistiken
sudo zpool iostat 1

# Verfügbarer Speicher
sudo zfs list
\end{verbatim}